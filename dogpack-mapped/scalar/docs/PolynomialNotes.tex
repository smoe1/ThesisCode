\documentclass[12pt]{amsart}
\begin{document}

\title{}
\author{}
\date{}
\maketitle

\section{4D Basis Functions}

The 4D basis functions up to second order will be given by
\begin{align}
	\phi^{(k)} \in \left\{	1, \xi, \eta, \mu, \tau \right\}.
\end{align}
Here, we use the convention that the usual mappings take place between
$(x,y,v_x, v_y) \leftrightarrow (\xi, \eta, \mu, \tau )$ for the canonical element.

The Cartesian basis functions (up to second order) are given by:
\begin{align}
	\phi^{(k)}_{2D_{CART}} \in \left\{ 1, \sqrt{3} \mu,  \sqrt{3} \tau \right\}.
\end{align}
These are orthogonal with respect to
\begin{align}
	\frac{1}{4} \iint_{[-1,1]}^{[-1,1]} \phi^{(k)}(\mu,\tau) \phi^{(\ell)}(\mu,\tau)\, d\mu\, d\tau = \delta_{k\ell}
\end{align}

The unstructured basis functions (up to second order) are given by:
\begin{align}
	\phi^{(k)}_{2D_{UNST}} \in \left\{ 1, 3\sqrt{2}\xi, \sqrt{6}\left( \xi + 2\eta \right) \right\}.
\end{align}
These are orthogonal with respect to,
\begin{align}
	2 \int_{-1/3}^{1/3}\int_{-1/3}^{1/3-\xi} \phi^{(k)}(\xi,\eta) \phi^{(\ell)}(\xi,\eta)\, d\eta\, d\xi = \delta_{k\ell}.
\end{align}
Therefore, the 4D basis functions (up to second order), are given by,
\begin{align}
	\phi \in \left\{ 1, 3\sqrt{2}\xi, \sqrt{6}\left( \xi + 2\eta \right), 	 \sqrt{3} \mu,  \sqrt{3} \tau	\right\}.
\end{align}
The third order terms introduce extra cross terms that need to be dealt with.
In the end, these terms are all orthogonal with respect to,
\begin{align}
	\frac{1}{2} \iint_{[-1,1]}^{[-1,1]} \int_{-1/3}^{1/3}\int_{-1/3}^{1/3-\xi} \phi^{(k)}(\xi,\eta, \mu, \tau) \phi^{(\ell)}(\xi,\eta, \mu, \tau)\, d\eta\, d\xi d\mu\, d\tau = \delta_{k\ell}.
\end{align}

\end{document}
